\section{Einführung in die Statistische Beschreibung}
\label{sec:StatistischeBeschreibung}
\subsection{Beispiele}
\subsection{Mathematische Beschreibung turbulenter Strömungen}
Die Strömung wird bei der Betrachtung als Kontinuum über die Erhaltungsgleichungen über die Zeit und Ort beschrieben. Mit der Massenerhaltung:
\begin{equation}
	\frac{D \rho}{D t}+\rho \frac{\partial u_{j}}{\partial x_{j}}=\frac{\partial \rho}{\partial t}+\rho \frac{\partial u_{j}}{\partial x_{j}}+u_{j} \frac{\partial \rho}{\partial x_{j}}=0
\end{equation}
als Summe der Substantiellen Ableitung der Dichte nach Ort und Zeit und der Örtlichen Ableitung der Geschwindigkeit. Alternativ kann:
\begin{equation}
	\frac{\partial \rho}{\partial t} + \vec{\nabla}(\rho \vec{u}) =  \frac{\partial \rho}{\partial t} + \vec{\nabla} \rho \vec{u} + \rho \vec{\nabla}\vec{u}
\end{equation}
mit $\nabla$ als die örtliche Ableitung. Die Impulserhaltung ergibt sich zu:
\begin{equation}
	\label{eq: ImpulsmitS_ij}
	\begin{aligned}
		\frac{\partial \rho u_{i}}{\partial t}+\frac{\partial \rho u_{j} u_{i}}{\partial x_{j}} &=-\frac{\partial p}{\partial x_{i}}+\frac{\partial \tau_{j i}}{\partial x_{j}}+\rho f_{i} \\
		&=-\frac{\partial p}{\partial x_{i}}+\frac{\partial}{\partial x_{j}}\left(2 \mu S_{i j}-\frac{2}{3} \mu \frac{\partial u_{k}}{\partial x_{k}} \delta_{i j}\right)+\rho f_{i}
	\end{aligned}
\end{equation}
Mit $\partial \rho u_{j} u_{i}/\partial x_i$ als dem konvektivem Term, $\tau_{ij}$ als die Scher oder Molekulare Spannung und einer Volumankraft. Wird der zweite Summand der Rechten Seite ausgeschrieben ergibt sich der untere Term. %TODO was ist hier genau passiert?
Dieser wird nur berechnet, wenn $i$ und $j$ gleich sind (Kronikadelta).
Unter der Annahme inkompressibler Strömung ($Ma <0,3$) wird $\rightarrow \rho = konst.$ Weiter wird keine Änderung der Temperatur angenommen und damit kann auch die Viskosität  als Konstant angenommen werden (Druckabhängigkeit der Viskosität um Skalen kleiner als die der Temperatur). Es folgt damit für die Massenerhaltung: 
\begin{equation}
	\label{eq: Massenerhaltung}
	\frac{\partial u_{j}}{\partial x_{j}}=0
\end{equation}
Damit kann die Strömung als Divergenzfrei beschrieben werden. Für die Impulserhaltung folgt nach einsetzen der $S_{ij}$ Matrix aus \ref{eq: ImpulsmitS_ij}: 
\begin{equation}
	\label{eq: Impulserhaltung}
	\rho \frac{Du_i}{Dt} = -\frac{\partial p}{\partial x_i}+\mu \frac{\partial^2 u_i}{\partial x^{2}_{i}} + \rho f_i
\end{equation}
Die Zeitliche Entwicklung der Strömung ist unter Angabe von Rand- und Anfangsbedingungen möglich. Damit folgt auch ein hoher Informationsgehalt (Eine Direkte Numerische Simulation [DNS] ist damit möglich, jedoch meist mit sehr viel Aufwand verbunden) 
Turbulente Strömung kann unterschiedlichste Skalen annehmen. Damit eine DNS durchgeführt werden kann müssen sowogl die Großen als auch die kleinen Instabilitäten aufgelöst werden. Das Verhältnis dieser Skalen kann über die Reynoldszahl abgeschätzt werden. 
\begin{equation}
	\frac{L}{\eta} \approx Re^{3/4} = \sqrt[4]{Re^3}
\end{equation}
Darin ist $L$ die größte Skala der Strömung und $\eta$ die kleinste. Auch die Zeitskala muss angepasst werden.
\begin{equation}
	\frac{t_L}{t_{\eta}} \approx Re^{1/2} = \sqrt{Re}
\end{equation}
Damit folgt für eine 3-Dimensionale Strömung, dass für eine Veränderte Strömung die Anzahl der Knoten in die Raumrichtungen sich wie folgt ändert:
\begin{equation}
	N_xN_yN_z \approx Re^{9/4}Re^{1/2} = Re^{11/4}
\end{equation}
Das bedeutet, dass beispielsweise eine Verdopplung der Strömungsgeschwindigkeit eine $\approx$8 -Fache Auflösung des Netzes erfordert. Aufgrund des schnell wachsenden aber auch schon für kleine Reynoldszahlen großen Aufwandes einer DNS werden diese nur verwendet, wenn es das Ziel ist fundamentale Strömungsphänomene zu beobachten oder auch Modelle herzuleiten, die das Phänomen abbilden können. Diese Modelle bilden die meist verwendeten Methoden zur Simulation einer Strömung. Dabei gibt es unterschiedlichste Ansätze. 


\subsection{Statistische Beschreibung}
Im folgenden betrachten wir häufig Mittellungen, die immer mit $\langle x \rangle_{i}$ beschrieben werden. Der Index gibt dabei an um welche Mittellung es sich handelt. 
Der Erwartungswert (Normiert und damit immer $1$):
\begin{equation}
	\langle\phi\rangle_{e}\left(x_{i}, t\right)=\int_{-\infty}^{\infty} \psi f_{\phi}\left(\psi ; x_{i}, t\right) d \psi = 1
\end{equation}
Ellemblemittelung über alle Realisierungen($\phi^{(n)}$):
\begin{equation}
	\label{eq: Essenmble}
	\langle\phi\rangle_{E}\left(x_{i}, t\right)=\lim _{N \rightarrow \infty} \frac{1}{N} \sum_{n=1}^{N} \phi^{(n)}\left(x_{i}, t\right)
\end{equation}
Zeitmittelung über alle Zeitlichen Realisierungen:
\begin{equation}
	\label{eq: Zeitmittel}
	\langle\phi\rangle_{t}\left(x_{i}\right)=\lim _{T \rightarrow \infty} \frac{1}{T} \int^{\frac{T}{2}}_{-\frac{T}{2}} \phi\left(x_{i}, t\right) d t
\end{equation}
Ortsmittel über alle örtlichen Realisierungen:
\begin{equation}
	\langle\phi\rangle_{x_{i}}(t)=\lim _{L \rightarrow \infty} \frac{1}{L} \int_{-\frac{L}{2}}^{\frac{L}{2}} \phi\left(x_{i}, t\right) d x_{i}
\end{equation}
Handelt es sich um eine statistische stationäre Strömung, kann das Ensemblemittel (Gleichung \ref{eq: Essenmble}) durch das Zeitmittel (Gleichung \ref{eq: Zeitmittel}) ersetzt werden (\textit{Ergodizität}).
Ist die Strömung räumlich homogen kann das Ensemblemittel durch das Ortsmittel ersetzt werden. Dies gilt auch bei partieller Homogenität (z.B. in einer Koordinatenrichtung). Diese Mittellung ist nicht erlaubt, wenn es in der Richtung Gradienten gibt. 


\subsubsection{Reynolds-Mittel}
Zu beginn einige Eigenschaften der Mittelungsoperationen.
\begin{equation*}
	\begin{aligned}
		\text { Projektion }\langle\langle\phi\rangle\rangle &=\langle\phi\rangle \\
		\text { Linearität }\langle\lambda \phi+\theta\rangle &=\lambda\langle\phi\rangle+\langle\theta\rangle \\
		\text { Konstanz des Mittels }\langle\phi\langle\theta\rangle\rangle &=\langle\phi\rangle\langle\theta\rangle
	\end{aligned}
\end{equation*}
Damit folgt für die Ableitung des Mittels:
\begin{equation}
	\left\langle\partial_{x_{i}} \phi\right\rangle=\partial_{x_{i}}\langle\phi\rangle \quad, \quad\left\langle\partial_{t} \phi\right\rangle=\partial_{t}\langle\phi\rangle
\end{equation}
Der Mittelwert der Schwanungen verschwindet und damit wird: 
\begin{equation}
	\left\langle\phi^{\prime}\right\rangle=0
\end{equation}
In der Reynoldschen Mittelung wird die Strömung in ihren Mittelwert und eine Schwankung um diesen Definiert. 
\begin{equation}
	\phi = \left\langle\phi\right\rangle + \phi^{\prime}
\end{equation}
Damit folgt mit der Massenerhaltung und Impilserhaltung (Gleichungen \ref{eq: Massenerhaltung} und \ref{eq: Impulserhaltung}) für die Reynolds-gemittelten Erhaltungsgleichungen:
\begin{equation}
	\begin{aligned}
		\frac{\partial\left\langle u_{j}\right\rangle}{\partial x_{j}}&=0 \\
		\frac{\partial \rho\left\langle u_{j}\right\rangle\left\langle u_{i}\right\rangle}{\partial x_{j}}&=-\frac{\partial\langle p\rangle}{\partial x_{i}}+\frac{\partial}{\partial x_{j}}\left(\mu \frac{\partial\left\langle u_{i}\right\rangle}{\partial x_{j}}-\rho\left\langle u_{i}^{\prime} u_{j}^{\prime}\right\}\right)+\left\langle F_{i}\right\rangle
	\end{aligned}
\end{equation}
%TODO Die Berechungen der Terme Nachvollziehen und selber rechnen. Folgen aus 3u 4 und den Mittelungsoperationen.
Dabei entstehen neue Unbekannte und wir bekommen ein unterbestimmtes System. 