\section{Lernziele}
\subsection{RANS Gleichungen}
\subsubsection{Phänomenologie}
\begin{enumerate}
	\item Was ist der Unterschied zwischen Laminarer und turbulenter Strömung?
	\item Nennen Sie Beispiele für verschiedene Strömungstypen.
	\item Erläutern Sie die Eigenschaften turbulenter Strömungen.
\end{enumerate}
\subsubsection{Mathematische Gleichungen}
\begin{enumerate}
	\item Nennen Sie die Gleichungen zur kontinuumsmechanischen Beschreibung von Strömungen und die Bedeutung der Terme. 
	\item Was ist die Skalenproblematik bei der genauen Beschreibung der turbulenz?
\end{enumerate}
\subsubsection{Statistische Beschreibung}
\begin{enumerate}
	\item Nennen Sie die Statistiken einer Stömungsgröße.
	\item Nennen sie die Mittelungsoperationen und ihre Eigenschaften. 
	\item Erläutern Sie die Herleitung der RANS Gleichungen und begründen Sie den Ursprung des Reynoldsspannungstensors (RST). 
	\item Nennen Sie den Unterschied zwischen RANS und URANS.
\end{enumerate}
\subsubsection{Herleitung RST Gleichungen}
\begin{enumerate}
	\item Was ist die Strategie zur Herleitung der Reynoldsspannungen?
	\item Nennen Sie die Terme der Gleichung für denRST  (Reynoldsspannungstensor).
	\item Was ist das Schließunsproblem der Turbulenz? 
\end{enumerate}


\subsection{02}